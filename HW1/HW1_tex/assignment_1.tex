%%%%%%%%%%%%%%%%%%%%%%%%%%%%%%%%%%%%%%%%%
% Short Sectioned Assignment
% LaTeX Template
% Version 1.0 (5/5/12)
%
% This template has been downloaded from:
% http://www.LaTeXTemplates.com
%
% Original author:
% Frits Wenneker (http://www.howtotex.com)
%
% License:
% CC BY-NC-SA 3.0 (http://creativecommons.org/licenses/by-nc-sa/3.0/)
%
%%%%%%%%%%%%%%%%%%%%%%%%%%%%%%%%%%%%%%%%%

%----------------------------------------------------------------------------------------
%	PACKAGES AND OTHER DOCUMENT CONFIGURATIONS
%----------------------------------------------------------------------------------------

\documentclass[paper=a4, fontsize=11pt]{scrartcl} % A4 paper and 11pt font size

\usepackage[T1]{fontenc} % Use 8-bit encoding that has 256 glyphs
%\usepackage{fourier} % Use the Adobe Utopia font for the document - comment this line to return to the LaTeX default
\usepackage[english]{babel} % English language/hyphenation
\usepackage{amsmath,amsfonts,amsthm} % Math packages
\usepackage{mathtools} %More math! (For dscases)
\usepackage{hyperref} %HTML package
\usepackage{pgfplots} %Makes plots in LaTeX
\usepackage{tikz} %Also tikz?
\usepackage{bm} %makes vectors bold
\usepackage{bbm} %Blackboard bold 1
\usepgfplotslibrary{fillbetween}%Let's me fill between named plots
\usepackage{graphicx} %import pics

\usepackage[noend]{algpseudocode} %Algorithms
\usepackage{algorithm}

\graphicspath{ {Python_figs/} }
\DeclareGraphicsExtensions{.pdf,.png,.jpg}
\usepackage{sectsty} % Allows customizing section commands
\allsectionsfont{ \normalfont\scshape} % Make all sections the default font and small caps


\renewcommand{\thesubsection}{\alph{subsection}} %Make subsections start with letters

\usepackage{fancyhdr} % Custom headers and footers
\pagestyle{fancyplain} % Makes all pages in the document conform to the custom headers and footers
\fancyhead{} % No page header - if you want one, create it in the same way as the footers below
\fancyfoot[L]{} % Empty left footer
\fancyfoot[C]{} % Empty center footer
\fancyfoot[R]{\thepage} % Page numbering for right footer
\renewcommand{\headrulewidth}{0pt} % Remove header underlines
\renewcommand{\footrulewidth}{0pt} % Remove footer underlines
\setlength{\headheight}{13.6pt} % Customize the height of the header

\numberwithin{equation}{section} % Number equations within sections (i.e. 1.1, 1.2, 2.1, 2.2 instead of 1, 2, 3, 4)
\numberwithin{figure}{section} % Number figures within sections (i.e. 1.1, 1.2, 2.1, 2.2 instead of 1, 2, 3, 4)
\numberwithin{table}{section} % Number tables within sections (i.e. 1.1, 1.2, 2.1, 2.2 instead of 1, 2, 3, 4)

\setlength\parindent{0pt} % Removes all indentation from paragraphs - comment this line for an assignment with lots of text

\usepackage{listings}
\lstset{language=Python}


%----------------------------------------------------------------------------------------
%	TITLE SECTION
%----------------------------------------------------------------------------------------

\newcommand{\horrule}[1]{\rule{\linewidth}{#1}} % Create horizontal rule command with 1 argument of height

\title{	Assignment 1}

\author{Benjamin Jakubowski} % Your name

\date{\normalsize\today} % Today's date or a custom date

\begin{document}

\maketitle % Print the title

%----------------------------------------------------------------------------------------
%	PROBLEM 1
%----------------------------------------------------------------------------------------

\section{Computing index of maximum element in array}

An algorithm to compute the index of the maximum element in an array is given below:

\begin{algorithm}
\caption{Index of maximum element in array}
\label{maxel}
\begin{algorithmic}[1]
\Function{Max\_index}{$A$}
	\State max\_val = $A[1]$
	\State max\_index = 1
	\For {j in $[2, \cdots, A.length]$}:
		\State candidate = $A$[j]
		\If {max\_val < candidate}:
			\State max\_val = candidate
			\State max\_index = j
		\EndIf
	\EndFor
	\Return max\_index
\EndFunction
\end{algorithmic}
\end{algorithm}

The loop invariant is as follows; at the start of the \textbf{for} loop of lines 4-8, max\_val is greater than or equal to all elements in the subarray $A[1, \cdots, j - 1]$. We now use this invariant to demonstrate the correctness of this algorithm.

\begin{itemize}
\item \textbf{Initialization}: First, we must show the loop invariant holds before the first loop iteration, when $j = 2$. The subarray $A[1, \cdots, j - 1] = A[1, \cdots, 2 - 1]$ is just the single element $A[1]$. Since max\_val was initialized to $A[1]$ in line 2, the loop invariant clearly holds.
\item \textbf{Maintainence}: Next, we must show that each iteration maintains the loop invariant. We have two cases, corresponding to the if-statement in line 4:
\begin{itemize}
\item \textbf{max\_val < candidate}: Then candidate must be greater than or equal to all the elements in the subarray $A[1, \cdots, j-1]$. Thus, setting max\_val to candidate in line 7 ensures max\_val is greater than or equal to all elements in the subarray $A[1, \cdots, j]$, maintaining the loop invariant before the start of the next iteration.
\item \textbf{max\_val $\geq$ candidate}: Then the if-block (lines 7-8) is not executed, and before the start of the next iteration max\_val is greater than or equal to all elements in the subarray $A[1, \cdots, j]$ (since it was assumed to be greater than or equal to all elements in $A[1, \cdots, j-1]$, as well as $A[j]$).
\end{itemize}
\item \textbf{Termination}: Finally, the loop terminates when $j = n + 1 > A.length = n$. Note at the start of this final iteration, max\_val is greater than or equal to all elements in $A[1, \cdots, j - 1] = A[1, \cdots, n + 1 - 1] = A[1, \cdots, n]$. Hence we conclude the algorithm finds the maximum value in the array $A$, and returns its corresponding index.
\end{itemize}

%----------------------------------------------------------------------------------------
%	PROBLEM 2
%----------------------------------------------------------------------------------------

\section{Lucas numbers}

The Lucas numbers as defined as follows:
\[
L_n =
\begin{cases}
2 & \qquad{} \textrm{if } n=0\\
1 & \qquad{} \textrm{if } n=1\\L
L_{n-1} + L+{n-2} & \qquad{} \textrm{if } n>1
\end{cases}
\]

We prove by induction the closed-form expression for the $n$-th Lucas number:
\[L_n = \varphi^n + (1-\varphi)^n\]
where $\varphi$ is the golden ratio:
\[\varphi = \frac{1 + \sqrt{5}}{2}\]

First, we prove two base cases:

\textbf{Base case 1}: Consider n=2. Then
\begin{align*}
L_{n = 2} &= L_{n - 1} + L_{n - 2} \\
	&= L_{2 - 1} + L_{2 - 2} \\
	&= L_{1} + L_{0} \\
	&= 1 + 2 = 3
\end{align*}
Also, 
\begin{align*}
\varphi^2 + (1-\varphi)^2 &= \varphi^2 + 1 - 2\varphi + \varphi^2 \\ 
	&= 2\varphi^2 - 2\varphi + 1 \\
	&= \frac{(1+\sqrt{5})^2}{2} - (1+\sqrt{5}) + 1 \\
	&= \frac{1+2\sqrt{5} +5}{2} - \sqrt{5} \\
	&= \frac{6}{2} = 3\\
\end{align*}

\textbf{Base case 2}: Consider n=3. Then
\begin{align*}
L_{n = 3} &= L_{3 - 1} + L_{3 - 2} \\
	&= L_{2} + L_{1} \\
	&= 3 + 1 = 4
\end{align*}
Also, 
\begin{align*}
\varphi^3 + (1-\varphi)^3 &= \varphi^3 + 1 - 3\varphi + 3\varphi^2 - \varphi^3 \\ 
	&= 1 - 3\varphi +3 \varphi^2 \\
	&= 1 - 3 \cdot \frac{1+\sqrt{5}}{2} + 3 \left(\frac{1+\sqrt{5}}{2}\right)^2 \\
	&= 1 - \frac{3}{2} - \frac{3\sqrt{5}}{2} +\frac{3}{4} + \frac{3\sqrt{5}}{2} + \frac{3\cdot5}{4} \\
	&= -\frac{1}{2} + \frac{18}{4} = 4\\
\end{align*}

Next, we proceed to the induction step. Assume
\[L_{n} = \varphi^n + (1 - \varphi)^n\]
\[L_{n+1} = \varphi^{n+1} + (1 - \varphi)^{n+1}\]
We need to show
\[L_{n+2} = \varphi^{n+2} + (1 - \varphi)^{n+2}\]

First, note
\begin{align*}
\varphi^2 &= \left(\frac{1 + \sqrt{5}}{2}\right)^2 \\
	&= \frac{1 + 2\sqrt{5}+5}{4} \\
	&= \frac{6 + 2\sqrt{5}+5}{4} \\
	&= \frac{3 + \sqrt{5}}{2} \\
	&= 1 + \frac{1 + \sqrt{5}}{2} \\
	&= 1 + \varphi
\end{align*}
Then
\begin{align*}
\varphi^{k+2} &= \varphi^k \cdot \varphi^2 \\
	&= \varphi^k \cdot (1 + \varphi) \\
	&= \varphi^k + \varphi^{k+1}
\end{align*}
Also, 
\begin{align*}
(1 - \varphi)^2 &= \left(1 - \frac{1 + \sqrt{5}}{2}\right)^2 \\
	&= 1 - 2\cdot \frac{1 + \sqrt{5}}{2} + \left(\frac{1 + \sqrt{5}}{2}\right)^2 \\
	&= 1 -1 - \sqrt{5} + \frac{1 + 2\sqrt{5}+5}{4} \\
	&= - \sqrt{5} + \frac{3}{2} + \frac{\sqrt{5}}{2} \\
	&= \frac{3- \sqrt{5}}{2} \\
	&= 2 - \frac{1- \sqrt{5}}{2} \\
	&= 1 + (1 - \varphi)
\end{align*}
So
\begin{align*}
(1 - \varphi)^{k+2} &= (1 - \varphi)^k (1 - \varphi)^2 \\
	&= (1 - \varphi)^k \left(1 + (1 - \varphi)\right) \\
	&= (1 - \varphi)^k +  (1 - \varphi)^{k+1}
\end{align*}

Now we put these two expressions together to complete the proof:

\begin{align*}
L_{n+2} &= L_{n+1} + L_{n} \\
	&= \varphi^{n+1} + (1 - \varphi)^{n+1} +  \varphi^{n} + (1 - \varphi)^{n}\\
	&= \varphi^{n+1}  +  \varphi^{n} + (1 - \varphi)^{n+1} + (1 - \varphi)^{n}\\
	&= \varphi^{n+2} + (1 - \varphi)^{n+2}
\end{align*}

completing the proof.

%----------------------------------------------------------------------------------------
%	PROBLEM 3
%----------------------------------------------------------------------------------------

\section{Properties of $\Theta$}
\subsection{Being in $\Theta$ is an equivalence relation}

To prove that being in $\Theta$ is an equivalence relation, we must show it is reflexive, symmetric, and transitive. We first show reflexivity.

Consider a function $f$, and assume it is asymptotically non-negative. Then let $c_1 = 1/2$, $c_2 = 3/2$, and $n_0$ be some constant such that $f(n) \geq 0$ for all $n > n_0$ (by asymptotic non-negativity). Then
\[0 \leq c_1 f(n) \leq f(n) \leq 3/2 f(n) \qquad{} \forall n \geq n_0\]
Thus, $f(n) \in \Theta(f(n))$.

Now we show symmetry- consider two functions $f$ and $g$. First, assume $f \in \Theta(g)$. Then there exists $c_1, c_2, n_0 > 0$ such that
\[0 \leq c_1 g(n) \leq f(n) \leq c_2g(n) \qquad{} \forall n > n_0\]
Then note
\[0 \leq g(n) \leq 1/c_1 f(n) \qquad{} \forall n > n_0\]
\[0 \leq 1/c_2 f(n) \leq g(n) \qquad{} \forall n > n_0\]
So, putting these inequalities together yeilds
\[0 \leq 1/c_2 f(n) \leq g(n) \leq 1/c_1 f(n) \qquad{} \forall n > n_0\]
Thus $g \in \Theta(f(n))$.

Now note the proof is essentially equivalent for the other direction (namely, assuming $g \in \Theta(f)$, that $g \in \Theta(f)$); all we'd do is switch the function names and use new constants. Thus this proof is omitted.

Finally, we show transitivity.

Assume $f \in \Theta(g)$ and $g \in \Theta(h)$. We show $f \in \Theta(h)$. Since $f \in \Theta(g)$, there exists $c_1, c_2, n_0 > 0$ such that
\[0 \leq c_1 g(n) \leq f(n) \leq c_2 g(n) \qquad{} \forall n > n_0\]
Since $g \in \Theta(h)$, there exists $c_3, c_4, n_1 > 0$ such that
\[0 \leq c_3 h(n) \leq g(n) \leq c_4 h(n) \qquad{} \forall n > n_1\]
But then substituting yields
\[0 \leq c_1\cdot c_3h(n) \leq f(n) \leq c_2 \cdot c_4 h(n) \qquad \forall n\geq \max\{n_0, n_1\}\]
Hence $f \in \Theta(h)$.

Thus, having shown reflexivity, symmetry, and transitivity, we've shown $\Theta$ is an equivalence relation.

\subsection{Maximum of two functions is in $\Theta$ of their sum}

Consider $f_1$ and $f_2$ (again assuming asymptotic non-negativity). We aim to show
\[\max \{f_1, f_2\} \in \Theta(f_1 + f_2)\]

First, let $f =\max\{f_1, f_2\}$ to easy notation. Then note
\[0 \leq f(n) \leq f_1(n) + f_2(n) \qquad \forall n > n_0\]
(again using asymptotic non-negativity to chose $n_0$ such that for all $n>n_0, f_1(n) \geq 0$ and $f_2(n) \geq 0$).

Moreover, note
\[0 \leq 1/2 \left(f_1(n) +f_2(n)\right) \leq f(n) \qquad{} \forall n>n_0\]
Since either $f(n) = f_1(n) > f_2(n)$ or $f(n) = f_2(n) > f_1(n)$. But then,
\[0 \leq 1/2 \left(f_1(n) +f_2(n)\right) \leq f(n) \leq f_1(n) + f_2(n) \qquad \forall n > n_0\]
Hence $f(n) = \max\{f_1(n), f_2(n)\} \in \Theta\left(f_1(n) + f_2(n)\right)$.


\subsection{Sum of two functions is in $\Theta$ of their maximum}

Now we show the sum of two functions is in $\Theta$ of their maximum in two lines!
\begin{enumerate}
\item Being in $\Theta$ is an equivalence relation, so it's symmetric.
\item Thus $\max\{f_1(n), f_2(n)\} \in \Theta\left(f_1(n) + f_2(n)\right) \implies \left(f_1(n) + f_2(n)\right) \in \Theta(\max\{f_1(n), f_2(n)\})$.
\end{enumerate}

%----------------------------------------------------------------------------------------
%	PROBLEM 4
%----------------------------------------------------------------------------------------

\section{Ranking function forms by order of growth}

The given forms are ranked in order of growth below (from slowest to fastest):
\begin{enumerate}
\item Constant
\item Logarithmic
\item Linear
\item Linearithmic
\item Polynomial
\item Exponential
\end{enumerate} 

To prove this ranking, we proceed pairwise. Throughout, recall $g(n) \in o(f(n))$ if
\[\lim_{n \to \infty} \frac{g(n)}{f(n)} = 0\]
Additionally, note we are treating logs as $\log_e$, since a change of basis only introduces a constant term. Additionally, we are ignoring all constants (since they don't change the limiting behavior). 

\subsubsection*{Constant vs. Logarithmic}
Let $g(n)=c$ and $f(n) = \log(n)$. Then
\begin{align*}
\lim_{n \to \infty} \frac{g(n)}{f(n)} &= \lim_{n \to \infty} \frac{c}{\log(n)} = 0
\end{align*}

\subsubsection*{Logarthimic vs. Linear}
Let $g(n)=\log(n)$ and $f(n) = n$. Then
\begin{align*}
\lim_{n \to \infty} \frac{g(n)}{f(n)} &= \lim_{n \to \infty} \frac{\log(n)}{n} \\
	&= \lim_{n \to \infty} \frac{(\log(n))'}{(n)'} \qquad{} \textrm{by l'Hopital's rule}\\
	&= \lim_{n \to \infty} \frac{1}{n} = 0
\end{align*}

\subsubsection*{Linear vs. Linearithmic}
Let $g(n)=n$ and $f(n) = n\log(n)$. Then
\begin{align*}
\lim_{n \to \infty} \frac{g(n)}{f(n)} &= \lim_{n \to \infty} \frac{n}{n\log(n)} \\
	&= \lim_{n \to \infty} \frac{1}{\log(n)} = 0
\end{align*}

\subsubsection*{Linearithmic vs. Polynomial}
Let $g(n)=n\log(n)$ and $f(n) = n^d$ for some $d>1$. Then
\begin{align*}
\lim_{n \to \infty} \frac{g(n)}{f(n)} &= \lim_{n \to \infty} \frac{n\log(n)}{n^d} \\
	&= \lim_{n \to \infty} \frac{\log(n)}{n^{d-1}} \\
	&= \lim_{n \to \infty} \frac{(\log(n))'}{(n^{d-1})'} \\
	&= \lim_{n \to \infty} \frac{n^{-1}}{(d-1)n^{d-2}} \\
	&= \lim_{n \to \infty} \frac{1}{d-1}n^{-1-(d-2)} \\
	&= \lim_{n \to \infty} \frac{1}{d-1}n^{1-d} = 0 \qquad{} \textrm{since d>1}\\
\end{align*}


\subsubsection*{Polynomial vs. Exponential}
Let $g(n)=n^d$ for some $ d > 1$ and $f(n) = a^n$ for some $a>1$. Then
\begin{align*}
\lim_{n \to \infty} \frac{g(n)}{f(n)} &= \lim_{n \to \infty} \frac{n^d}{a^n} \\
	&= \lim_{n \to \infty} \frac{(n^d)'}{(a^n)'} \\
	&= \lim_{n \to \infty} \frac{d \cdot n^{d-1}}{\log(a) a^{n}} \\
	& \qquad{}  \vdots \qquad{} \lceil d \rceil \textrm{ more applications of l'Hopital's Rule} \\
	&= \lim_{n \to \infty} \frac{k\cdot n^{d - \lceil d \rceil - 1}}{\log(a)^{\lceil d \rceil+1}a^{n}} \qquad{} \textrm{where } k \textrm{ is just a constant} \\
	&= 0	
\end{align*}

%----------------------------------------------------------------------------------------

\end{document}